\documentclass[]{article}
\usepackage{setspace}
\usepackage{lmodern}
\usepackage{amssymb,amsmath}
\usepackage{ifxetex,ifluatex}
\usepackage{listings}
\usepackage{fixltx2e} % provides \textsubscript
\ifnum 0\ifxetex 1\fi\ifluatex 1\fi=0 % if pdftex
  \usepackage[T1]{fontenc}
  \usepackage[utf8]{inputenc}
\else % if luatex or xelatex
  \ifxetex
    \usepackage{mathspec}
  \else
    \usepackage{fontspec}
  \fi
  \defaultfontfeatures{Ligatures=TeX,Scale=MatchLowercase}
  \newcommand{\euro}{€}
\fi
% use upquote if available, for straight quotes in verbatim environments
\IfFileExists{upquote.sty}{\usepackage{upquote}}{}
% use microtype if available
\IfFileExists{microtype.sty}{%
\usepackage{microtype}
\UseMicrotypeSet[protrusion]{basicmath} % disable protrusion for tt fonts
}{}
\usepackage[margin=1in]{geometry}
\usepackage{hyperref}
\PassOptionsToPackage{usenames,dvipsnames}{color} % color is loaded by hyperref
\hypersetup{unicode=true,
            pdftitle={Untitled},
            pdfauthor={Andre Guimaraes Duarte},
            pdfborder={0 0 0},
            breaklinks=true}
\urlstyle{same}  % don't use monospace font for urls
\usepackage{color}
\usepackage{fancyvrb}
\newcommand{\VerbBar}{|}
\newcommand{\VERB}{\Verb[commandchars=\\\{\}]}
\DefineVerbatimEnvironment{Highlighting}{Verbatim}{commandchars=\\\{\}}
% Add ',fontsize=\small' for more characters per line
\usepackage{framed}
\definecolor{shadecolor}{RGB}{248,248,248}
\newenvironment{Shaded}{\begin{snugshade}}{\end{snugshade}}
\newcommand{\KeywordTok}[1]{\textcolor[rgb]{0.13,0.29,0.53}{\textbf{{#1}}}}
\newcommand{\DataTypeTok}[1]{\textcolor[rgb]{0.13,0.29,0.53}{{#1}}}
\newcommand{\DecValTok}[1]{\textcolor[rgb]{0.00,0.00,0.81}{{#1}}}
\newcommand{\BaseNTok}[1]{\textcolor[rgb]{0.00,0.00,0.81}{{#1}}}
\newcommand{\FloatTok}[1]{\textcolor[rgb]{0.00,0.00,0.81}{{#1}}}
\newcommand{\ConstantTok}[1]{\textcolor[rgb]{0.00,0.00,0.00}{{#1}}}
\newcommand{\CharTok}[1]{\textcolor[rgb]{0.31,0.60,0.02}{{#1}}}
\newcommand{\SpecialCharTok}[1]{\textcolor[rgb]{0.00,0.00,0.00}{{#1}}}
\newcommand{\StringTok}[1]{\textcolor[rgb]{0.31,0.60,0.02}{{#1}}}
\newcommand{\VerbatimStringTok}[1]{\textcolor[rgb]{0.31,0.60,0.02}{{#1}}}
\newcommand{\SpecialStringTok}[1]{\textcolor[rgb]{0.31,0.60,0.02}{{#1}}}
\newcommand{\ImportTok}[1]{{#1}}
\newcommand{\CommentTok}[1]{\textcolor[rgb]{0.56,0.35,0.01}{\textit{{#1}}}}
\newcommand{\DocumentationTok}[1]{\textcolor[rgb]{0.56,0.35,0.01}{\textbf{\textit{{#1}}}}}
\newcommand{\AnnotationTok}[1]{\textcolor[rgb]{0.56,0.35,0.01}{\textbf{\textit{{#1}}}}}
\newcommand{\CommentVarTok}[1]{\textcolor[rgb]{0.56,0.35,0.01}{\textbf{\textit{{#1}}}}}
\newcommand{\OtherTok}[1]{\textcolor[rgb]{0.56,0.35,0.01}{{#1}}}
\newcommand{\FunctionTok}[1]{\textcolor[rgb]{0.00,0.00,0.00}{{#1}}}
\newcommand{\VariableTok}[1]{\textcolor[rgb]{0.00,0.00,0.00}{{#1}}}
\newcommand{\ControlFlowTok}[1]{\textcolor[rgb]{0.13,0.29,0.53}{\textbf{{#1}}}}
\newcommand{\OperatorTok}[1]{\textcolor[rgb]{0.81,0.36,0.00}{\textbf{{#1}}}}
\newcommand{\BuiltInTok}[1]{{#1}}
\newcommand{\ExtensionTok}[1]{{#1}}
\newcommand{\PreprocessorTok}[1]{\textcolor[rgb]{0.56,0.35,0.01}{\textit{{#1}}}}
\newcommand{\AttributeTok}[1]{\textcolor[rgb]{0.77,0.63,0.00}{{#1}}}
\newcommand{\RegionMarkerTok}[1]{{#1}}
\newcommand{\InformationTok}[1]{\textcolor[rgb]{0.56,0.35,0.01}{\textbf{\textit{{#1}}}}}
\newcommand{\WarningTok}[1]{\textcolor[rgb]{0.56,0.35,0.01}{\textbf{\textit{{#1}}}}}
\newcommand{\AlertTok}[1]{\textcolor[rgb]{0.94,0.16,0.16}{{#1}}}
\newcommand{\ErrorTok}[1]{\textcolor[rgb]{0.64,0.00,0.00}{\textbf{{#1}}}}
\newcommand{\NormalTok}[1]{{#1}}
\usepackage{graphicx,grffile}
\makeatletter
\def\maxwidth{\ifdim\Gin@nat@width>\linewidth\linewidth\else\Gin@nat@width\fi}
\def\maxheight{\ifdim\Gin@nat@height>\textheight\textheight\else\Gin@nat@height\fi}
\makeatother
% Scale images if necessary, so that they will not overflow the page
% margins by default, and it is still possible to overwrite the defaults
% using explicit options in \includegraphics[width, height, ...]{}
\setkeys{Gin}{width=\maxwidth,height=\maxheight,keepaspectratio}
\setlength{\parindent}{0pt}
\setlength{\parskip}{6pt plus 2pt minus 1pt}
\setlength{\emergencystretch}{3em}  % prevent overfull lines
\providecommand{\tightlist}{%
  \setlength{\itemsep}{0pt}\setlength{\parskip}{0pt}}
\setcounter{secnumdepth}{0}

%%% Use protect on footnotes to avoid problems with footnotes in titles
\let\rmarkdownfootnote\footnote%
\def\footnote{\protect\rmarkdownfootnote}

%%% Change title format to be more compact
\usepackage{titling}

% Create subtitle command for use in maketitle
\newcommand{\subtitle}[1]{
  \posttitle{
    \begin{center}\large#1\end{center}
    }
}

\setlength{\droptitle}{-2em}
  \title{MSAN 601 - Pitches}
  \pretitle{\vspace{\droptitle}\centering\huge}
  \posttitle{\par}
  \author{Andre Guimaraes Duarte}
  \preauthor{\centering\large\emph}
  \postauthor{\par}
  \predate{\centering\large\emph}
  \postdate{\par}
  \date{August 30, 2016}

\doublespacing

% Redefines (sub)paragraphs to behave more like sections
\ifx\paragraph\undefined\else
\let\oldparagraph\paragraph
\renewcommand{\paragraph}[1]{\oldparagraph{#1}\mbox{}}
\fi
\ifx\subparagraph\undefined\else
\let\oldsubparagraph\subparagraph
\renewcommand{\subparagraph}[1]{\oldsubparagraph{#1}\mbox{}}
\fi

\begin{document}
\maketitle

%For this assignment, please answer the following questions using informal, oral language.
%Since Pitches are spoken, not written, the rules of grammar are a bit relaxed. That being
%said, I expect these to be well-edited and free of unintentional grammar errors. Each of
%these questions should be answered around 30 seconds, make sure to time yourself before
%submitting.

\subsection*{The Ice Breaker}
%You have joined a hot new internet start-up called “Meow-Meet.”
%Think Tinder for cat-lovers. At your first all hands meeting, the CEO introduces you
%and asks you to talk a little bit about yourself.
Hey everyone. I'm Andre. I just joined the Data Science team here at Meow-Meet, and I'm very excited to start working on great analytics projects with you! I just graduated from a masters in Analytics, so I look forward to applying all the skills I've learned in school here. I will make time this week to introduce myself properly to each of you, or you can of course come talk to me. Thanks.

\subsection*{Dream Job}
%You have finally landed your dream job, working in the data science
%group for the city of Net York. On your first day, your boss introduces you to the
%Mayor, who is making a quick morning round. She asks “What brought you to this
%job?”
Hi, my name is Andre. I just graduated from a masters in Analytics, where I learned a lot about the tools and skills used in data science. I always wanted to join a team of really talented and diverse individuals who are working hard towards making a meaningful difference in the world. This was always my first choice, and I am really happy to be working here with this group. I look forward to start contributing to some great projects here.

\subsection*{The Interview (part 1)}
%During a culture fit interview, you are asked the following
%question: “Why did you decide to go to the USF MSAN program?”
There are a few key factors that went into my decision to go to the USF MSAN program. The fact that they have a required internship with many partners in the industry was definitely a great incentive. USF has had this program for five years now, and they've been constantly updating the curriculum to the demands of the industry, adding new courses as they become more relevant in the job market. This really showed me that they understand the industry more than some other programs that I was interested in.

\subsection*{The Interview (part 2)}
%During a culture fit interview, you are asked the following
%question: “What about data science interests you?”
I find it fascinating the amount of data that we've been generating since the past few years. We now have connected everything: watches, shoes, coffee-makers, cars... and all of these are producing data. So now we have all this data, but we need to extract the meaningful information out of it, we need to make sense of it. I think it's really interesting to think about how to do this, how to extract actual information out of huge amount of messy data. That to me is the most incredible thing about data science.

\subsection*{Introduce Yourself}
%You have joined a hot new internet start-up and on the first
%day of work you are asked to “Introduce yourself and tell us something that makes
%you unique.”
Hey everyone. I'm Andre. I just joined the Data Science team here, and I'm really excited to start working on fun analytics projects! A fun fact about me: I practiced Capoeira, the Brazilian martial art/dance, for about five years until I broke my foot doing an acrobatic stunt. Now I'm looking for a place to pick it back up, so if anyone knows of a gym that has classes, or if anyone is interested in either Capoeira or other martial arts, let's talk! Thanks.

\newpage
\subsection*{Personal Narrative}
%The final part of the assignment is for you to write-up a one page personal narrative
%answering the following question: “Why are you here?”
I went to university at one of the best engineering schools in France. After two years of basic/general engineering, I decided to specialize in bioinformatics and modeling. Although biology was never my true calling, I was mostly interested in seeing how to apply computer science and mathematics in a specific field, instead of studying "pure" computer science or mathematics. During those three years, I learned how to program in Python, delved into statistical analysis in R, and had plenty of exposure to linear algebra. For my final internship before graduating, I went to the City University of New York to do research in bone mechanics. My role at this lab was to write an R script (and a couple helper Python scripts) to treat and analyze stress-strain curves from bones automatically. By the end of my stay, all my co-workers were using my code. However, during my year there, I slowly realized that it was not exactly what I wanted to do. I enjoyed some parts of the job, but I wasn't happy.

I started looking at the job market, and I realized that most of the roles I was applying for were in Data Science. Although my background was not exactly that, I still had some of the skills required (or so I thought). My school in France had given my tools, but not the knowledge on how to use them in order to build something. Therefore, I decided to get a second masters degree, this time in Data Science, in order to learn those missing skills (in addition to getting a better understanding of those I had already).

I got my first acceptance letter from a university in England, shortly followed by a couple in Australia. But I really wanted to stay in the US. Out of all my choices, USF stood out as my top one. The practicum was a deciding factor to me, as it would be the perfect way to get actual work experience while in the program. This was also the only program that lasts one year instead of the usual two, and since I had just spent five years in school in France, I appreciated this. Finally, the geography was a key factor in my decision: being in the heart of Silicon Valley is definitely something that few univerisites can boast about.

I had my interview with Nathaniel, and it went great. I shortly afterwards got an email from Kirsten and David saying that I had been accepted, and was even offered a partial scholarship. My decision had been made, and now here I am. Based on bootcamp and this first week of classes, I can confidently say that I made the right choice. I feel excited as I haven't felt in a long while, and I am more than motivated to make the most of it.

\end{document}
