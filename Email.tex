\documentclass[]{article}
\usepackage{lmodern}
\usepackage{setspace}
\usepackage{graphicx}
\usepackage{adjustbox}
\usepackage{amssymb,amsmath}
\usepackage{ifxetex,ifluatex}
\usepackage{listings}
\usepackage{indentfirst}
\usepackage{color}
  \usepackage[T1]{fontenc}
  \usepackage[utf8]{inputenc}
\usepackage{microtype}
\usepackage[margin=1in]{geometry}
\usepackage{hyperref}
\usepackage{framed}
\usepackage{graphicx,grffile}
\usepackage[colorinlistoftodos]{todonotes}
\makeatletter
\def\maxwidth{\ifdim\Gin@nat@width>\linewidth\linewidth\else\Gin@nat@width\fi}
\def\maxheight{\ifdim\Gin@nat@height>\textheight\textheight\else\Gin@nat@height\fi}
\makeatother
% Scale images if necessary, so that they will not overflow the page
% margins by default, and it is still possible to overwrite the defaults
% using explicit options in \includegraphics[width, height, ...]{}
\setkeys{Gin}{width=\maxwidth,height=\maxheight,keepaspectratio}
%\setlength{\parindent}{0pt}
\setlength{\parskip}{6pt plus 2pt minus 1pt}
\setlength{\emergencystretch}{3em}  % prevent overfull lines
\providecommand{\tightlist}{%
  \setlength{\itemsep}{0pt}\setlength{\parskip}{0pt}}

%%% Change title format to be more compact
\usepackage{titling}

% Create subtitle command for use in maketitle
\newcommand{\subtitle}[1]{
  \posttitle{
    \begin{center}\large#1\end{center}
    }
}

\setlength{\droptitle}{-2em}
  \title{MSAN 610 - Email}
  \pretitle{\vspace{\droptitle}\centering\huge}
  \posttitle{\par}
  \author{Arda Aysu, Andre Guimaraes Duarte, Rui Li, Anshika Srivastava}
  \preauthor{\centering\large\emph}
  \postauthor{\par}
  \predate{\centering\large\emph}
  \postdate{\par}
  \date{September 13, 2016}
  
% Redefines (sub)paragraphs to behave more like section*s
\ifx\paragraph\undefined\else
\let\oldparagraph\paragraph
\renewcommand{\paragraph}[1]{\oldparagraph{#1}\mbox{}}
\fi
\ifx\subparagraph\undefined\else
\let\oldsubparagraph\subparagraph
\renewcommand{\subparagraph}[1]{\oldsubparagraph{#1}\mbox{}}
\fi

\doublespacing

%%%%%%%%%%%%%%%%%%%%%%%%%%%%%%%%%%%%%%%%%%%%%%%%%%%%%%%%%%%%%%%%%%%%%%%%%%%%%%%%%%%%%%%%%%%%%%%%%%%%%%%%%%%%%%%%%%%%%%%
\begin{document}
\maketitle

\section*{Getting Dragged into a Conversation}
\subsection*{Email}
\noindent Subject: RE: FW: Click-through Rates \\
To: John@zynga.com \\
CC:  kim@zynga.com, benny@zynga.com, Hunter@zynga.com \\
John,

I think there has been a big misunderstanding. Let me attempt to provide some clarity as to what happened.

While running some test queries on our database, I found it interesting that cross-promotional advertisements in our games had half the click-through rates of non-Zynga advertisements. This was only a casual observation of a single metric, as I am sure that many other factors go into optimizing advertising decisions. At no point did I intend to undermine the work of any department at Zynga.

I can expand on the details if needed. Sorry for all the confusion.

-(name)

\subsection*{Discussion}
The above email is intended to be apologetic, where an individual takes the blame as a subordinate employee. We wanted to make sure the recipients understood the truth about the situation: it was simply an observation that was in no way meant to undermine the work of another team. However, we thought we should not be overly apologetic either, as that would convey an idea of having done something wrong, which was not the case. In addition, we made sure to write the email in first person, so that it would not be interpreted as you pointing the finger and shifting the blame to a coworker (which would have been more negative).

Another goal of ours was to use a writing style that would allow John to directly forward this email to Kathy if he felt it was appropriate. This saves time and allows for more transparent communication. It also reduced the risk of John getting a negative impression of "me", as he does not have extra work in writing an email to Kathy himself. An alternative strategy would have been to talk to John directly, but that would have meant additional load on John since he would then have to contact Kathy or write an email himself.

We decided a short and to-the-point email was the best approach in order to imply that this is not a big issue, and was in fact only a misunderstanding that got blown out of proportion. By keeping the email short and not going too much into detail, we convey this idea.

By ending the email with an open invitation to give additional details, we allow Kathleen to come forward if she wants to further investigate our finding. The idea is that she does not feel belittled by the situation.

\section*{PIP}
\subsection*{Email}
\noindent Subject: PIP for Rebecca \\
To: Rebecca \\
BCC: Tom \\
Rebecca,

As you are aware, Tom and I are implementing a Personal Improvement Plan for you to follow. Below are the key areas in which we would like to see improvement:

\begin{itemize}
\item Responsible behavior at company social events, e.g. happy hours;

\item On-time attendance to all meetings;

\item More constructive attitude towards company and product direction;

\item Equally-high standards for your work on all projects.
\end{itemize}

I will meet with you later today to further discuss how you can achieve these goals. I am always available to answer any questions or concerns you may have.

-(name)

\subsection*{Discussion}
For this email, we decided to keep it simple as the person would go into deeper details during the 1-on-1 meeting with Rebecca. Since Tom is BCC'd and has previously talked with "me" and her, we thought it would be better if the email was kept short and objective.

We opted in listing only the ways in which we expect Rebecca to improve. There is no need to include the reasons why she has been placed in a PIP, as all concerned parties are aware of them already. Again, more detail is expected to be given when the two meet later in the day.

The tone of the email is strict and blunt to demonstrate that we will not be accepting any kind of misbehavior from Rebecca's part anymore. However, we also do not want her to feel attacked and undermined, so we felt it necessary to end the email on a positive note. After all, she can be a great asset to the team as long as she works hard and her behavior improves. Therefore, we want to seem open to discussion and ready to help her through this process. Achieving the right balance of sternness and kindness was a big difficulty in this exercise.

\end{document}