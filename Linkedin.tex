\documentclass[]{article}
\usepackage{lmodern}
\usepackage{setspace}
\usepackage{graphicx}
\usepackage{adjustbox}
\usepackage{amssymb,amsmath}
\usepackage{ifxetex,ifluatex}
\usepackage{listings}
\usepackage{indentfirst}
\usepackage{color}
  \usepackage[T1]{fontenc}
  \usepackage[utf8]{inputenc}
\usepackage{microtype}
\usepackage[margin=1in]{geometry}
\usepackage{hyperref}
\usepackage{framed}
\usepackage{graphicx,grffile}
\makeatletter
\def\maxwidth{\ifdim\Gin@nat@width>\linewidth\linewidth\else\Gin@nat@width\fi}
\def\maxheight{\ifdim\Gin@nat@height>\textheight\textheight\else\Gin@nat@height\fi}
\makeatother
% Scale images if necessary, so that they will not overflow the page
% margins by default, and it is still possible to overwrite the defaults
% using explicit options in \includegraphics[width, height, ...]{}
\setkeys{Gin}{width=\maxwidth,height=\maxheight,keepaspectratio}
%\setlength{\parindent}{0pt}
\setlength{\parskip}{6pt plus 2pt minus 1pt}
\setlength{\emergencystretch}{3em}  % prevent overfull lines
\providecommand{\tightlist}{%
  \setlength{\itemsep}{0pt}\setlength{\parskip}{0pt}}

%%% Change title format to be more compact
\usepackage{titling}

% Create subtitle command for use in maketitle
\newcommand{\subtitle}[1]{
  \posttitle{
    \begin{center}\large#1\end{center}
    }
}

\setlength{\droptitle}{-2em}
  \title{MSAN 610 - LinkedIn}
  \pretitle{\vspace{\droptitle}\centering\huge}
  \posttitle{\par}
  \author{Arda Aysu, Andre Guimaraes Duarte, Rui Li, Anshika Srivastava}
  \preauthor{\centering\large\emph}
  \postauthor{\par}
  \predate{\centering\large\emph}
  \postdate{\par}
  \date{September 6, 2016}
  
% Redefines (sub)paragraphs to behave more like section*s
\ifx\paragraph\undefined\else
\let\oldparagraph\paragraph
\renewcommand{\paragraph}[1]{\oldparagraph{#1}\mbox{}}
\fi
\ifx\subparagraph\undefined\else
\let\oldsubparagraph\subparagraph
\renewcommand{\subparagraph}[1]{\oldsubparagraph{#1}\mbox{}}
\fi

\doublespacing

%%%%%%%%%%%%%%%%%%%%%%%%%%%%%%%%%%%%%%%%%%%%%%%%%%%%%%%%%%%%%%%%%%%%%%%%%%%%%%%%%%%%%%%%%%%%%%%%%%%%%%%%%%%%%%%%%%%%%%%
\begin{document}
\maketitle

\section*{Hardest profile to finish}
We found Rui's profile to be the most difficult to finish. Under \textit{Education}, Rui previously had a very detailed explanation of her Master at the University of East Anglia, but no details under the Master at USF or her Bachelor at Beijing Jiaotong University. The result of this was an unbalanced Education section that over-emphasized one education experience. We had to rephrase the previous description such as to keep all the relevant information while keeping it short (from seven to two lines).

In addition, we also had to rewrite Rui's \textit{Experience} entry at United Boron Energy Materials S\&T LLC. She had a link to the company website as well as an uploaded image file of the company logo, but we deleted them since they are irrelevant information. The description of her work duties were lacking active verbs and were very long. Through rewording the text, we managed to condense the information into only three lines (from seven).

\section*{Easiest profile to finish}
We found Arda's profile to be the easiest to finish since most details were already filled out and well-written. All that was left to do was trim the excess fat. This was done in two ways: removing unnecessary information and condensing work experience.

Information such as birthday, interests, and undergraduate concentrations was removed. These are either no longer relevant or potentially discouraging for recruiters who may not see their field listed under his interests. Arda's more recent work experience was changed into bullets for readability purposes. His older internships were combined and simplified so that they did not overshadow his longer and more recent work experience.

\section*{Discussions}
\paragraph{Condensing data}
Regarding the long expressions and block texts, we decided to condense the data in order to achieve a cleaner look. We tried to split long sentences and descriptions into bullet points, and added action verbs for each of them. We also joined some recurring themes and ideas under individual entries. All of us condensed data on our profile in descriptions of \textit{Education}, \textit{Experience}, and some in \textit{Summary}. In general, we thought that having shorter and more direct descriptions was preferable to having long chunks of text, especially when the information was repetitive.

\paragraph{Removing past experiences not related to Analytics}
There were multiple instances of work experiences that were not relevant to Analytics in our profiles. We decided to either reduce or remove them as they might distract from our Analytics education, exposure, and work experience. Andre and Rui ended up shortening their \textit{Experience} sections as a result.

\paragraph{Degree mapping}
We found problems in mapping  Anshika and Andre's undergraduate degrees, since their diploma titles do not match the American BS and MS system exactly. Anshika has a four year engineering degree which is written as \textit{Bachelor of Technology}. Since that might not make sense to most of the recruiters and it is equivalent to Bachelor of Science, we changed it to \texttt{Bachelor of Science (BS)}. Andre's diploma is a five-year Engineering Diploma which includes a Bachelor and Master (but with no distinction between them), and we decided that the best option was to write it as \texttt{Master of Science (BS+MS)} to show that both BS and MS were acquired. 

\paragraph{Skills}
We found that many of our listed skills were not very relevant in the field of Analytics. To correct this, we removed the unrelated skills without many endorsements from our lists. We also moved the \textit{Skills} section to the back of our profiles, since it is underdeveloped at this time.

\paragraph{Others}
We found that most of us did not have particularly good profile photos. We all agreed on updating them as soon as our professional shots taken at USF a few weeks ago are released. We also improved our personal \textit{url}s. Finally, we reordered the sections and agreed to keep the \textit{Education} section on the top, seeing as it is the most recent experience we have.

\end{document}
